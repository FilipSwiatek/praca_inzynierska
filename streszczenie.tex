\chapter{Streszczenie pracy}
Praca polegała na opracowaniu i wykonaniu układu multipleksera SWD - urządzenia pozwalającego na sekwencyjne programowanie pamięci  do ośmiu mikrokontrolerów poprzez interfejs SWD przy użyciu tylko jednego programatora.\\
Urządzenia docelowe muszą posiadać wsparcie dla interfejsu SWD oraz wtyczkę w standardzie CoreSight10, lub WSN Connector.\\
Programator musi działać w standardzie ARM JTAG 20.
Praca multipleksera SWD jest być kontrolowana przez program oparty o język Python 3.\\ 
Program ten wysyła komendy  i odbiera potwierdzenia przez port szeregowy.\\
Używanie multipleksera SWD ma na celu:
\begin{enumerate}
    \item uproszczenie stanowiska testowego
    \item zmniejszenie kosztów stanowiska testowego
    \item dodanie funkcji jednoczesnego resetu urządzeń docelowych
    \item zautomatyzowanie procesu wgrywania oprogramowania do pamięci urządzeń docelowych
    \item eliminację wpływu czynnika ludzkiego podczas używania platformy testowej
\end{enumerate}

Wykonanie projektu zakończyło się sukcesem w przeważającej części. Nie działa jedynie dodatkowa komunikacja z multiplekserem SWD poprzez wbudowany w J-Linka (programator) UART.
