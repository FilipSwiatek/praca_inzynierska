\chapter{Słowniczek pojęć}
Lista ważnych, używanych w pracy pojęć:
\begin{enumerate}
\item 
\textbf{GDB Serwer} -  program komputerowy, który pośredniczy pomiędzy programem GNU Debugger a systemem docelowym (urządzeniem docelowym) W przypadku tej pracy program zarządza programatorem komunikującym się z urządzeniem docelowym
\item 
\textbf{SWD} - Swrial Wire Debug - interfejs komunikacyjny pomiędzy programatorem i mikrokontrolerem z rdzeniem ARM Cortex-M
\item 
\textbf{interfejs} - wspólne połączenie, poprzez które dwa lub więcej oddzielnych elementów wymienia się informacjami w ściśle określony sposób.
\item 
\textbf{urządzenie docelowe} - w przypadku pracy urządzenie oparte na mikrokontrolerze z rdzeniem ARM Cortex-M.
\item 
\textbf{mikrokontroler} – układ cyfrowy z wyspecjalizowanym  mikroprocesorem i niezbędnymi do  jego samodzielnej pracy urządzeniami zawartymi w jednym układzie scalonym. Źródło:\cite{mikrokontroler_definicja}\\
\item
\textbf{USB} - Uniwersal Serial Bus - interfejs uniwersalnej magistrali szeregowej ustandaryzowany przez USB Implementers Forum
\item
\textbf{multiplekser} - Przełącznik zrealizowany w postaci układu elektronicznego. Najczęściej jest to układ scalony. Wyprowadzenie o adresie wybranym na złączach sterujących połączone jest z wyprowadzeniem bez adresu. W przypadku multipleksera analogowego nie rozróżnia się wejść ani wyjść, ponieważ połączenie jest dwukierunkowe.
\item 
\textbf{programator} - urządzenie obsługujące interfejsy wymagane do zaprogramowania pamięci. Najczęściej używa się ich do programowania mikrokontrolerów. Najczęściej programatory wspierają również interfejsy diagnostyczne służące do  programowania i nadzorowania pracy urządzeń w systemie.
\item
\textbf{daisy chain} - Połączenie szeregowe urządzeń w topologii pierścienia. Polega na połączeniu wyjścia jednego urządzenia do wejścia drugiego.
\item
\textbf{UART} - nadajnik i odbiornik asynchronicznej transmisji szeregowej.

    
\end{enumerate}
