\chapter{Podsumowanie}

W ramach pracy wykonano następujące czynności:
\begin{enumerate}
    \item Zaprojektowanie i wykonanie sprzętowe część projektu.\\
    Projekt wykonano w programie Altium Designer 19 przy użyciu biblioteki elementów elektronicznych przygotowanych w zespole Bezprzewodowych Sieci Kontrolno-Pomiarowych Katedry Elektroniki AGH. W czasie trwania projektu rozbudowano także wspomnianą bibliotekę o nowe podzespoły, których wcześniej nie zawierała. 
    \item Opracowanie oprogramowania układowego multipleksera SWD.\\ Oprogramowanie jest łatwo przenośne pomiędzy rodziną mikrokontrolerów STM32 zawierających interfejs USB, ponieważ użyto w nim bibliotek HAL producenta.
    \item Przygotowanie oprogramowania sterującego multiplekserem SWD z poziomu komputera z systemem operacyjnym Windows.\\
    Oprogramowanie to zawiera kontrolę błędów zarówno po stronie interfejsu użytkownika jak i multipleksera SWD. Pomaga to wykryć np błędy w połączeniu multipleksera SWD, urządzeń docelowych lub programatora. W prosty sposób można dodać funkcjonalność w postaci obsługi innych programatorów i wsparcie pozostałych systemów operacyjnych wspieranych przez środowisko Python 3.
    \item Testy działania urządzenia pod kontrolą komputera.\\
    Próbie został podjęty zarówno multiplekser SWD jak i program kontrolujący urządzenie z poziomu komputera klasy PC.
\end{enumerate}

Płytka PCB została wykonana w fabryce na podstawie projektu wyeksportowanego do plików formatu gerber, a jej wymiary to \SI{112.78}{mm} x \SI{51.8}{mm}.
Posiada otwory montażowe umożliwiające prosty montaż płyty w obudowie.

Zdecydowaną zaletą urządzenia jest uproszczenie i minimalizacja kosztów produkcji i testowania oprogramowania systemów wbudowanych zawierających interfejs SWD.

Praca opublikowana zostanie jako otwartoźródłowy projekt na platformie GitHub.


