\chapter{Wstęp}
Celem pracy było zaprojektowanie i wykonanie multipleksera SWD (Serial Wire Debug) - urządzenia, które na podstawie komend przełącza linie sygnałowe interfejsu SWD pomiędzy programatorem a ośmioma urządzeniami docelowymi. Urządzenie powinno być sterowane i zasilane z USB, jednak dopuszcza się zasilanie jak i sterowanie poprzez interfejsy programatora.

\section{Potrzeba stworzenia urządzenia przełączającego}
Celem stworzenia multipleksera SWD było uproszczenie i automatyzacja procesu testowania oprogramowania dla urządzeń zbudowanych w oparciu o mikrokontroler. Ze względu na dużą popularność mikrokontrolerów wykorzystujących rdzenie z rodziny ARM Cortex-M, zdecydowano się na wykorzystanie interfejsu SWD (ang. Serial Wire Debug). Do jego zalet zaliczyć można szybkość programowania układów oraz małą liczbę wykorzystywanych wyprowadzeń. Wadą SWD względem popularnego w przeszłości standardu JTAG (ang. Joint Test Action Group) jest brak możliwości spięcia układów szeregowo w tzw. łańcuch (ang daisy chain), co pozwalało na programowanie wielu układów bez konieczności przepinania programatora. Brak tej funkcjonalności jest uciążliwy i Wiąże się z dużym problemem podczas programowania kilku urządzeń w krótkim odstępie czasu. \\
Programowanie wielu urządzeń na raz można zrealizować na 2 sposoby:
\begin{enumerate}
    \item Zastosowanie osobnego programatora dla każdego urządzenia docelowego. Wiąże się to ze znacznymi kosztami. Istnieje potrzeba zastosowania dużej ilości programatorów  (koszt jednego to nierzadko kilkaset a nawet kilka tysięcy PLN). Programatory te najczęściej są podłączane do komputera poprzez interfejs USB, a zatem w tym przypadku wymagane jest tyle samo portów USB co urządzeń docelowych. W przypadku indywidualnego projektu, małej lub nawet średniej firmy, koszty takiej automatyzacji stają się zbyt wysokie względem korzyści odniesionych w procesie tworzenia urządzeń docelowych.
    \item Przełączanie jednego programatora pomiędzy poszczególne urządzenia docelowe. Rozwiązanie znacznie tańsze, jednakże dla zwiększenia niezawodności poprzez ograniczenie wpływu zużycia gniazd i wtyczek, oraz ograniczenia wpływu błędu ludzkiego, należy zautomatyzować proces poprzez przełączanie linii sygnałowych pomiędzy urządzenia docelowe za pomocą dodatkowego urządzenia. Takim urządzeniem jest właśnie multiplekser SWD.
\end{enumerate}

Bezsprzecznie lepszym wyborem jest wybranie drugiej opcji ze względu na znacznie niższe koszty implementacji i znacznie mniej miejsca zajmowanego na stanowisku testowym. Wadami takiego rozwiązania jest czas poświęcony na opracowanie, wykonanie i testowanie urządzenia przełączającego, oraz fakt, że pomaga on jedynie w przypadku urządzeń wykorzystujących interfejs SWD.

\section{Prace wykonane w ramach projektu}
Prace w ramach projektu obejmowały:
\begin{enumerate}
    \item Weryfikację wstępną wykonalności projektu w rozsądnym czasie
    \item Wykonanie schematu blokowego urządzenia
    \item Dobór elementów ze względu na charakterystykę interfejsu SWD oraz możliwie jak najmniejszy rozmiar urządzenia
    \item Utworzenie schematu ideowego sprzętu
    \item Wykonanie projektu płytki drukowanej
    \item Napisanie oprogramowania układowego
    \item Stworzenie oprogramowania sterującego na PC
    \item Montaż elementów elektronicznych na płytce drukowanej
    \item Uruchomienie i sprawdzenie poprawności połączeń
    \item Końcowe testy oprogramowania i sprzętu
\end{enumerate}


